\documentclass[a4,9pt]{beamer}
\usetheme{Berlin}
\usepackage{multicol}
\usepackage{xcolor}
\usepackage{graphicx}
\linespread{1.35}
\usepackage{amsmath}
\usepackage{color}
\usepackage{tikz}
\usetikzlibrary{arrows,automata}

\begin{document}

\begin{frame}
\section*{Minimization of Finite Automata}
\begin{flushright}
 \texttt{} \hspace*{0.1cm}\textbf{$|$} \hspace*{0.1cm} \textbf{27}\hspace*{0.1cm}
\end{flushright}
\vspace*{1cm}
CHAPTER THREE
DESIGNING e-RESEARCH
It is the theory decides what can be observed. Albert Einstein
Because the net is a large, multipurpose, evolving tool, determining its best use in
any research application is a challenging task. however, the net is also famous for
spurring innovation at”Internet speed”, frequently leaving authors of paper books
struggling ti keep up. in this chapter we discuss what is perhaps the most important
and challenging task of the e-researcher– to design research that asks meaningful and
answerable questions and that coherently, the sponsor of the research, and the subjects
of investigation.
\end{frame}
\begin{document}

\begin{frame}
\section*{Minimization of Finite Automata}
\begin{flushright}
 \texttt{} \hspace*{0.1cm}\textbf{$|$} \hspace*{0.1cm} \textbf{27}\hspace*{0.1cm}
\end{flushright}
\vspace*{1cm}

Considerable research is being conducted using the internet as a data-gathering,
analysis, and dissemination tool, even though the advantages and disadvantages of
using the internet for these purposes remain relatively unexplored. often, those using
the net do so with little guidance with respect to what kind of research data is most
appropriately collected online. based on work by early adopters of e-research, it would
appear that when the researcher has a good understanding of the net(including its culture
and technological limitation and advantages) that almost any kind of research
could be effectively adapted. further, when creatively approached and thoughtfully
designed, research can be conducted and disseminated using the net with a number of
notable advantages, which are discussed in the last section of this chapter. 
\end{frame}
\begin{document}

\begin{frame}
\section*{Minimization of Finite Automata}
\begin{flushright}
 \texttt{} \hspace*{0.1cm}\textbf{$|$} \hspace*{0.1cm} \textbf{27}\hspace*{0.1cm}
\end{flushright}
\vspace*{1cm}

this being
said, thee are circumstances under which the net will be of little or no use to the
research process. at one time, for example, the net was only useful for observing
activities that took place on it. now, however, net- based surveys, focus groups ,interview
data about events that take place both on and off the net. much of the research in the social
sciences and education focuses on processesthat cannot ba seen and measured with external
and quantifiable tools(e.g.,the internal mental processes of learning). Since these processes are invisible,
it take the innovative skills of the researcher to develop both net and non-net techniques to understand

\end{frame}
\end{document}